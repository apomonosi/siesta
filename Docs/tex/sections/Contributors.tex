The SIESTA project was initiated by Pablo Ordej\'on (then at the Univ. de
Oviedo), and Jos\'e M. Soler and Emilio Artacho (Univ. Aut\'onoma de Madrid,
UAM).  The development team was then joined by Alberto Garc\'ia (then at
Univ. del Pa\'is Vasco, Bilbao), Daniel S\'anchez-Portal (UAM), and
Javier Junquera (Univ. de Oviedo and later UAM), and sometime later by
Julian Gale (then at Imperial College, London). In 2007 Jos\'e M. Cela
(Barcelona Supercomputing Center, BSC) became a core developer and
member of the Steering Committee.

The original \tsiesta\ module was developed by Pablo Ordej\'on and Jos\'e
L. Mozos (then at ICMAB-CSIC), and Mads Brandbyge, Kurt Stokbro, and
Jeremy Taylor (Technical Univ. of Denmark, DTU).

The current \tsiesta\ module within SIESTA is developed by Nick
R. Papior and Mads Brandbyge (DTU). Nick R. Papior became a core
developer and member of the Steering Committee in 2015.

Other contributors (we apologize for any omissions):

Anthoni Alcaraz,
Eduardo Anglada,
Thomas Archer,
Luis C. Balb\'as,
Laura Bellentani,
Xavier Blase,
Jorge I. Cerd\'a,
Ram\'on Cuadrado,
Michele Ceriotti,
Fabiano Corsetti,
Raul de la Cruz,
Jos\'e Mar\'ia Escart\'in,
Gabriel Fabricius,
Pol Febrer,
Mariv\'i Fern\'andez-Serra,
Jaime Ferrer,
Chu-Chun Fu,
Sandra Garc\'ia,
V\'ictor M. Garc\'ia-Su\'arez,
Rogeli Grima,
Julio Guti\'errez,
Xu He,
Rainer Hoft,
Georg Huhs,
Jorge Kohanoff,
Arnold H. Kole,
Peter Koval,
Richard Korytar,
In-Ho Lee,
Lin Lin,
Nicol\'as Lorente,
Miquel Llunell,
Eduardo Machado,
Maider Machado,
Jos\'e Lu\'is Martins,
Volodymyr Maslyuk,
Sara G. Mayo,
Juana Moreno,
Stephan Mohr,
Frederico Dutilh Novaes,
Micael Oliveira,
Magnus Paulsson,
\'Oscar Paz,
Federico Pedron,
Andrei Postnikov,
Yann Pouillon,
J. Miguel A. Pruneda,
Roberto Robles,
Tristana Sondon,
Rafi Ullah,
Andrew Walker,
Andrew Walkingshaw,
Toby White,
Fran\c{c}ois Willaime,
Nils Wittemeier,
Chao Yang,
Victor Yu.

O.F. Sankey, D.J. Niklewski and D.A. Drabold made the FIREBALL code
available to P. Ordej\'on.  Although we no longer use the routines in
that code, it was essential in the initial development of SIESTA,
which still uses many of the algorithms developed by them.
